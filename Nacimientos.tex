% Options for packages loaded elsewhere
\PassOptionsToPackage{unicode}{hyperref}
\PassOptionsToPackage{hyphens}{url}
%
\documentclass[
]{article}
\usepackage{amsmath,amssymb}
\usepackage{lmodern}
\usepackage{iftex}
\ifPDFTeX
  \usepackage[T1]{fontenc}
  \usepackage[utf8]{inputenc}
  \usepackage{textcomp} % provide euro and other symbols
\else % if luatex or xetex
  \usepackage{unicode-math}
  \defaultfontfeatures{Scale=MatchLowercase}
  \defaultfontfeatures[\rmfamily]{Ligatures=TeX,Scale=1}
\fi
% Use upquote if available, for straight quotes in verbatim environments
\IfFileExists{upquote.sty}{\usepackage{upquote}}{}
\IfFileExists{microtype.sty}{% use microtype if available
  \usepackage[]{microtype}
  \UseMicrotypeSet[protrusion]{basicmath} % disable protrusion for tt fonts
}{}
\makeatletter
\@ifundefined{KOMAClassName}{% if non-KOMA class
  \IfFileExists{parskip.sty}{%
    \usepackage{parskip}
  }{% else
    \setlength{\parindent}{0pt}
    \setlength{\parskip}{6pt plus 2pt minus 1pt}}
}{% if KOMA class
  \KOMAoptions{parskip=half}}
\makeatother
\usepackage{xcolor}
\IfFileExists{xurl.sty}{\usepackage{xurl}}{} % add URL line breaks if available
\IfFileExists{bookmark.sty}{\usepackage{bookmark}}{\usepackage{hyperref}}
\hypersetup{
  pdftitle={NacimientosFlorida},
  pdfauthor={Nacho Zobian Massetti},
  hidelinks,
  pdfcreator={LaTeX via pandoc}}
\urlstyle{same} % disable monospaced font for URLs
\usepackage[margin=1in]{geometry}
\usepackage{color}
\usepackage{fancyvrb}
\newcommand{\VerbBar}{|}
\newcommand{\VERB}{\Verb[commandchars=\\\{\}]}
\DefineVerbatimEnvironment{Highlighting}{Verbatim}{commandchars=\\\{\}}
% Add ',fontsize=\small' for more characters per line
\usepackage{framed}
\definecolor{shadecolor}{RGB}{248,248,248}
\newenvironment{Shaded}{\begin{snugshade}}{\end{snugshade}}
\newcommand{\AlertTok}[1]{\textcolor[rgb]{0.94,0.16,0.16}{#1}}
\newcommand{\AnnotationTok}[1]{\textcolor[rgb]{0.56,0.35,0.01}{\textbf{\textit{#1}}}}
\newcommand{\AttributeTok}[1]{\textcolor[rgb]{0.77,0.63,0.00}{#1}}
\newcommand{\BaseNTok}[1]{\textcolor[rgb]{0.00,0.00,0.81}{#1}}
\newcommand{\BuiltInTok}[1]{#1}
\newcommand{\CharTok}[1]{\textcolor[rgb]{0.31,0.60,0.02}{#1}}
\newcommand{\CommentTok}[1]{\textcolor[rgb]{0.56,0.35,0.01}{\textit{#1}}}
\newcommand{\CommentVarTok}[1]{\textcolor[rgb]{0.56,0.35,0.01}{\textbf{\textit{#1}}}}
\newcommand{\ConstantTok}[1]{\textcolor[rgb]{0.00,0.00,0.00}{#1}}
\newcommand{\ControlFlowTok}[1]{\textcolor[rgb]{0.13,0.29,0.53}{\textbf{#1}}}
\newcommand{\DataTypeTok}[1]{\textcolor[rgb]{0.13,0.29,0.53}{#1}}
\newcommand{\DecValTok}[1]{\textcolor[rgb]{0.00,0.00,0.81}{#1}}
\newcommand{\DocumentationTok}[1]{\textcolor[rgb]{0.56,0.35,0.01}{\textbf{\textit{#1}}}}
\newcommand{\ErrorTok}[1]{\textcolor[rgb]{0.64,0.00,0.00}{\textbf{#1}}}
\newcommand{\ExtensionTok}[1]{#1}
\newcommand{\FloatTok}[1]{\textcolor[rgb]{0.00,0.00,0.81}{#1}}
\newcommand{\FunctionTok}[1]{\textcolor[rgb]{0.00,0.00,0.00}{#1}}
\newcommand{\ImportTok}[1]{#1}
\newcommand{\InformationTok}[1]{\textcolor[rgb]{0.56,0.35,0.01}{\textbf{\textit{#1}}}}
\newcommand{\KeywordTok}[1]{\textcolor[rgb]{0.13,0.29,0.53}{\textbf{#1}}}
\newcommand{\NormalTok}[1]{#1}
\newcommand{\OperatorTok}[1]{\textcolor[rgb]{0.81,0.36,0.00}{\textbf{#1}}}
\newcommand{\OtherTok}[1]{\textcolor[rgb]{0.56,0.35,0.01}{#1}}
\newcommand{\PreprocessorTok}[1]{\textcolor[rgb]{0.56,0.35,0.01}{\textit{#1}}}
\newcommand{\RegionMarkerTok}[1]{#1}
\newcommand{\SpecialCharTok}[1]{\textcolor[rgb]{0.00,0.00,0.00}{#1}}
\newcommand{\SpecialStringTok}[1]{\textcolor[rgb]{0.31,0.60,0.02}{#1}}
\newcommand{\StringTok}[1]{\textcolor[rgb]{0.31,0.60,0.02}{#1}}
\newcommand{\VariableTok}[1]{\textcolor[rgb]{0.00,0.00,0.00}{#1}}
\newcommand{\VerbatimStringTok}[1]{\textcolor[rgb]{0.31,0.60,0.02}{#1}}
\newcommand{\WarningTok}[1]{\textcolor[rgb]{0.56,0.35,0.01}{\textbf{\textit{#1}}}}
\usepackage{graphicx}
\makeatletter
\def\maxwidth{\ifdim\Gin@nat@width>\linewidth\linewidth\else\Gin@nat@width\fi}
\def\maxheight{\ifdim\Gin@nat@height>\textheight\textheight\else\Gin@nat@height\fi}
\makeatother
% Scale images if necessary, so that they will not overflow the page
% margins by default, and it is still possible to overwrite the defaults
% using explicit options in \includegraphics[width, height, ...]{}
\setkeys{Gin}{width=\maxwidth,height=\maxheight,keepaspectratio}
% Set default figure placement to htbp
\makeatletter
\def\fps@figure{htbp}
\makeatother
\setlength{\emergencystretch}{3em} % prevent overfull lines
\providecommand{\tightlist}{%
  \setlength{\itemsep}{0pt}\setlength{\parskip}{0pt}}
\setcounter{secnumdepth}{-\maxdimen} % remove section numbering
\ifLuaTeX
  \usepackage{selnolig}  % disable illegal ligatures
\fi

\title{NacimientosFlorida}
\author{Nacho Zobian Massetti}
\date{2023-02-21}

\begin{document}
\maketitle

\hypertarget{natalidad-en-el-estado-de-florida}{%
\subsection{Natalidad en el estado de
Florida}\label{natalidad-en-el-estado-de-florida}}

En la base de datos de Nac.csv se encuentran los datos de los
nacimientos por dia del año en cada uno de los distintos estados de
estados unidos. El reto es, en primer lugar determinar si el numero de
nacimiento en los dias de luna llena en el estado de Florida (FL)
difiere de alguna forma con el resto de días.

\begin{Shaded}
\begin{Highlighting}[]
\CommentTok{\#Leemos el fichero Nac.csv}
\NormalTok{nac }\OtherTok{=} \FunctionTok{read.csv}\NormalTok{(}\StringTok{"Nac.csv"}\NormalTok{, }\AttributeTok{header =} \ConstantTok{FALSE}\NormalTok{, }\AttributeTok{sep=}\StringTok{";"}\NormalTok{)}
\CommentTok{\#Cabeceras}
\FunctionTok{names}\NormalTok{(nac) }\OtherTok{=} \FunctionTok{c}\NormalTok{(}\StringTok{"Numero\_de\_orden"}\NormalTok{, }\StringTok{"Estado"}\NormalTok{, }\StringTok{"Año"}\NormalTok{, }\StringTok{"Mes"}\NormalTok{, }\StringTok{"Dia"}\NormalTok{, }\StringTok{"Fecha"}\NormalTok{,}
               \StringTok{"Dia\_de\_la\_semana"}\NormalTok{, }\StringTok{"Numero\_de\_nacimientos"}\NormalTok{)}
\CommentTok{\#Primeros 100 Estados}
\NormalTok{Estados }\OtherTok{=} \FunctionTok{unique}\NormalTok{(nac}\SpecialCharTok{$}\NormalTok{Estado)}
\CommentTok{\#Florida (FL) es el estado numero 10 en orden alfabético}
\NormalTok{Estados[}\DecValTok{10}\NormalTok{]}
\end{Highlighting}
\end{Shaded}

\begin{verbatim}
## [1] "FL"
\end{verbatim}

El estudio ha de hacerse exclusivamente en el estado de florida por lo
que no tiene sentido mantener toda la base de datos con los demás
estados

Además, solo necesitamos la fecha y el numero de nacimientos en cada día
por lo que filtramos también (función select) por estos dos parámetros.

\begin{Shaded}
\begin{Highlighting}[]
\NormalTok{nacFL }\OtherTok{=} \FunctionTok{filter}\NormalTok{(nac, Estado}\SpecialCharTok{==}\StringTok{"FL"}\NormalTok{) }\CommentTok{\#Filtramos la información}
\NormalTok{nacFLr }\OtherTok{=} \FunctionTok{select}\NormalTok{(nacFL, Fecha, Numero\_de\_nacimientos)}
\end{Highlighting}
\end{Shaded}

La base de datos contiene los nacimientos desde el 1/1/1969 Se necesita
tener identificados que días del ciclo lunar corresponden a qué días del
calendario -\textgreater{} Para ello, y teniendo en cuenta que el
periodo en días del ciclo lunar no es un número entero, se usa un
equivalente a la función módulo que hace uso de ``floor'' (aunque en
realidad podría también utilizar ``round''). El hecho de que el periodo
del ciclo lunar sea de 29.53 conlleva necesariamente que haya ciclos de
30 días de duración y otros de tan solo 29. Hemos de tener en cuenta
esto último para la interpretación de los resultados.

\begin{Shaded}
\begin{Highlighting}[]
\NormalTok{dias}\OtherTok{=}\FunctionTok{as.numeric}\NormalTok{(}\FunctionTok{as.Date}\NormalTok{(nacFLr}\SpecialCharTok{$}\NormalTok{Fecha,}\StringTok{"\%Y{-}\%m{-}\%d"}\NormalTok{)}\SpecialCharTok{{-}}\FunctionTok{as.Date}\NormalTok{(}\StringTok{"1969{-}01{-}01"}\NormalTok{,}\StringTok{"\%Y{-}\%m{-}\%d"}\NormalTok{))}
\CommentTok{\#Añadimos la columna dias al dataframe}
\NormalTok{nacFLrd}\OtherTok{=}\FunctionTok{cbind}\NormalTok{(nacFLr,dias)}
\CommentTok{\#Calcular el día del ciclo lunar sabiendo que el 1{-}1{-}1969 era el 12avo dia del ciclo}
\NormalTok{dia\_ciclo\_lunar}\OtherTok{=}\FunctionTok{as.numeric}\NormalTok{(}\FunctionTok{floor}\NormalTok{((nacFLrd}\SpecialCharTok{$}\NormalTok{dias}\SpecialCharTok{+}\DecValTok{12}\NormalTok{)}\SpecialCharTok{{-}}\FunctionTok{floor}\NormalTok{((nacFLrd}\SpecialCharTok{$}\NormalTok{dias}\SpecialCharTok{+}\DecValTok{12}\NormalTok{)}\SpecialCharTok{/}\FloatTok{29.53}\NormalTok{)}\SpecialCharTok{*}\FloatTok{29.53}\NormalTok{))}
\NormalTok{nacFLrdl }\OtherTok{=} \FunctionTok{cbind}\NormalTok{(nacFLrd, dia\_ciclo\_lunar)}
\end{Highlighting}
\end{Shaded}

\hypertarget{recta-de-regresiuxf3n}{%
\subsection{Recta de Regresión}\label{recta-de-regresiuxf3n}}

Calculamos la recta de regresión del Número de nacimientos sobre el dia
del ciclo lunar para ver si existe algún tipo de relación lineal entre
las dos variables

\begin{Shaded}
\begin{Highlighting}[]
\FunctionTok{lm}\NormalTok{(nacFLrdl}\SpecialCharTok{$}\NormalTok{Numero\_de\_nacimientos}\SpecialCharTok{\textasciitilde{}}\NormalTok{nacFLrdl}\SpecialCharTok{$}\NormalTok{dia\_ciclo\_lunar)}
\end{Highlighting}
\end{Shaded}

\begin{verbatim}
## 
## Call:
## lm(formula = nacFLrdl$Numero_de_nacimientos ~ nacFLrdl$dia_ciclo_lunar)
## 
## Coefficients:
##              (Intercept)  nacFLrdl$dia_ciclo_lunar  
##                303.30686                   0.05078
\end{verbatim}

\begin{Shaded}
\begin{Highlighting}[]
\FunctionTok{plot}\NormalTok{(nacFLrdl}\SpecialCharTok{$}\NormalTok{dia\_ciclo\_lunar, nacFLrdl}\SpecialCharTok{$}\NormalTok{Numero\_de\_nacimientos)}
\FunctionTok{abline}\NormalTok{(}\FunctionTok{lm}\NormalTok{(nacFLrdl}\SpecialCharTok{$}\NormalTok{Numero\_de\_nacimientos}\SpecialCharTok{\textasciitilde{}}\NormalTok{nacFLrdl}\SpecialCharTok{$}\NormalTok{dia\_ciclo\_lunar))}
\end{Highlighting}
\end{Shaded}

\includegraphics{Nacimientos_files/figure-latex/unnamed-chunk-5-1.pdf}

Viendo la gráfica superior podemos empezar a sacar conclusiones: no
parece haber ninguna relacion entre el dia del ciclo lunar y el numero
de nacimientos en el estado de Florida o al menos esa relación no es
lineal. Esta conclusión se enfatiza si vemos el coeficiente de
correlación de Pearson.

\begin{Shaded}
\begin{Highlighting}[]
\NormalTok{Pearson }\OtherTok{=} \FunctionTok{cor}\NormalTok{(nacFLrdl}\SpecialCharTok{$}\NormalTok{Numero\_de\_nacimientos,nacFLrdl}\SpecialCharTok{$}\NormalTok{dia\_ciclo\_lunar)}
\NormalTok{Pearson}
\end{Highlighting}
\end{Shaded}

\begin{verbatim}
## [1] 0.01029156
\end{verbatim}

Una forma de llegar a conclusiones de una forma más rápida incluso sería
mirar el total de nacimientos por día del ciclo lunar y analizar si hay
algún día que resalte sobre el resto.

\begin{Shaded}
\begin{Highlighting}[]
\CommentTok{\# Vector de sumas}
\NormalTok{acumdcl}\OtherTok{=}\DecValTok{1}\SpecialCharTok{:}\DecValTok{30}
\ControlFlowTok{for}\NormalTok{(i }\ControlFlowTok{in} \DecValTok{1}\SpecialCharTok{:}\DecValTok{30}\NormalTok{)\{acumdcl[i]}\OtherTok{=}\FunctionTok{sum}\NormalTok{(nacFLrdl}\SpecialCharTok{$}\NormalTok{Numero\_de\_nacimientos[}\FunctionTok{which}\NormalTok{(nacFLrdl}\SpecialCharTok{$}\NormalTok{dia\_ciclo\_lunar}\SpecialCharTok{==}\NormalTok{i}\DecValTok{{-}1}\NormalTok{)])\}}
\FunctionTok{barplot}\NormalTok{(acumdcl)}
\end{Highlighting}
\end{Shaded}

\includegraphics{Nacimientos_files/figure-latex/unnamed-chunk-7-1.pdf}

\hypertarget{por-quuxe9-hay-un-duxeda-distinto-al-resto-tienen-razuxf3n-los-muxe9dicos-y-es-verdad-que-nacen-menos-niuxf1os-los-duxedas-30-de-cada-ciclo-lunar}{%
\subsubsection{Por qué hay un día distinto al resto? Tienen razón los
médicos y es verdad que nacen menos niños los días 30 de cada ciclo
lunar?}\label{por-quuxe9-hay-un-duxeda-distinto-al-resto-tienen-razuxf3n-los-muxe9dicos-y-es-verdad-que-nacen-menos-niuxf1os-los-duxedas-30-de-cada-ciclo-lunar}}

Es cierto, hay un dia (el 30 de cada ciclo lunar) que tiene
significativamente menos nacimientos. Pero ello no se debe a que esos
dias nazcan menos niños sino que tiene que ver con el hecho de que el
periodo del ciclo lunar no es un número entero por lo que no todos los
ciclos lunares ocupan 30 días naturales.

\#Estudio con el día de la semana. El 1/1/1969 fue miércoles. Sabemos
que en 1 semana hay 7 días por lo que aplicaremos una operación modular
para añadir una columna que indique (siendo 0 el lunes y 6 el domingo)
que día de la semana corresponde a cada día del calendario desde 1969.

\begin{Shaded}
\begin{Highlighting}[]
\CommentTok{\#Calcular el día del ciclo lunar sabiendo que el 1{-}1{-}1969 era el 12avo dia del ciclo}
\NormalTok{dia\_semana}\OtherTok{=}\FunctionTok{as.numeric}\NormalTok{(}\FunctionTok{floor}\NormalTok{((nacFLrd}\SpecialCharTok{$}\NormalTok{dias}\SpecialCharTok{+}\DecValTok{2}\NormalTok{)}\SpecialCharTok{{-}}\FunctionTok{floor}\NormalTok{((nacFLrd}\SpecialCharTok{$}\NormalTok{dias}\SpecialCharTok{+}\DecValTok{2}\NormalTok{)}\SpecialCharTok{/}\DecValTok{7}\NormalTok{)}\SpecialCharTok{*}\DecValTok{7}\NormalTok{))}
\NormalTok{nacFLrdl }\OtherTok{=} \FunctionTok{cbind}\NormalTok{(nacFLrd, dia\_semana)}
\end{Highlighting}
\end{Shaded}

La hipótesis de la que partimos es que el día de la semana \textbf{NO}
afecta al número de nacimientos. De ser así debería haberse visto
reflejado en el apartado anterior puesto que, en cada ciclo lunar hay
29.53/7 = 4.21 semanas. Si hubiese algunos días en los que hay más
nacimientos que otros veríamos aproximadamente unos 4 periodos de una
función periodica de periodo 7 al representar el barplot anterior y no
parece ser el caso. Tampoco podemos apresurarnos y tomar
precipitadamente un veredicto con esta primera intuición pero parece ser
un buen punto de partida.

\begin{Shaded}
\begin{Highlighting}[]
\FunctionTok{plot}\NormalTok{(nacFLrdl}\SpecialCharTok{$}\NormalTok{dia\_semana, nacFLrdl}\SpecialCharTok{$}\NormalTok{Numero\_de\_nacimientos)}
\FunctionTok{abline}\NormalTok{(}\FunctionTok{lm}\NormalTok{(nacFLrdl}\SpecialCharTok{$}\NormalTok{Numero\_de\_nacimientos}\SpecialCharTok{\textasciitilde{}}\NormalTok{nacFLrdl}\SpecialCharTok{$}\NormalTok{dia\_semana))}
\end{Highlighting}
\end{Shaded}

\includegraphics{Nacimientos_files/figure-latex/unnamed-chunk-9-1.pdf}

Sin embargo, cuando se dibuja la recta de regresión podemos observar
como parece haber cierta tendencia a haber menos nacimientos hacia el
final de la semana. Comprobemos el coeficiente de linealidad de Pearson.

\begin{Shaded}
\begin{Highlighting}[]
\NormalTok{Pearson }\OtherTok{=} \FunctionTok{cor}\NormalTok{(nacFLrdl}\SpecialCharTok{$}\NormalTok{Numero\_de\_nacimientos,nacFLrdl}\SpecialCharTok{$}\NormalTok{dia\_semana)}
\NormalTok{Pearson}
\end{Highlighting}
\end{Shaded}

\begin{verbatim}
## [1] -0.3874783
\end{verbatim}

Un coeficiente de linealidad de Pearson de -0.38 indica una correlación
negativa. Y, aunque si bien es verdad que hay una aparente tendencia
contraria de las dos variables el coeficiente de linealidad no parece
ser lo suficientemente elevado (en valor absoluto) como para derrumbar
nuestra hipótesis.

\begin{Shaded}
\begin{Highlighting}[]
\NormalTok{acumdcl}\OtherTok{=}\DecValTok{1}\SpecialCharTok{:}\DecValTok{7}
\ControlFlowTok{for}\NormalTok{(i }\ControlFlowTok{in} \DecValTok{1}\SpecialCharTok{:}\DecValTok{7}\NormalTok{)\{acumdcl[i]}\OtherTok{=}\FunctionTok{sum}\NormalTok{(nacFLrdl}\SpecialCharTok{$}\NormalTok{Numero\_de\_nacimientos[}\FunctionTok{which}\NormalTok{(nacFLrdl}\SpecialCharTok{$}\NormalTok{dia\_semana}\SpecialCharTok{==}\NormalTok{i}\DecValTok{{-}1}\NormalTok{)])\}}
\FunctionTok{barplot}\NormalTok{(acumdcl, }\AttributeTok{name =} \FunctionTok{c}\NormalTok{(}\StringTok{"Lunes"}\NormalTok{, }\StringTok{"Martes"}\NormalTok{, }\StringTok{"Miercoles"}\NormalTok{, }\StringTok{"Jueves"}\NormalTok{, }\StringTok{"Viernes"}\NormalTok{,}\StringTok{"Sabado"}\NormalTok{, }\StringTok{"Domingo"}\NormalTok{), }\AttributeTok{col =} \StringTok{"red"}\NormalTok{)}
\end{Highlighting}
\end{Shaded}

\includegraphics{Nacimientos_files/figure-latex/unnamed-chunk-11-1.pdf}

De nuevo se muestra lo observado en la gráfica anterior. Pareciera que
los sábados y domíngos se tiene un menor número de nacimientos que el
resto de días. En este punto parece que nuestra hipótesis empieza a
tambalearse puesto que el número de niños que nacieron en un domingo fue
de 48723 mientras que en viernes fueron 57101 niños. Una diferencia de
9429 niños.

Haciendo un poco de investigación encontramos los siguientes dos
artículos:

\begin{enumerate}
\def\labelenumi{\arabic{enumi}.}
\tightlist
\item
  \url{https://www.lasexta.com/programas/lasexta-clave/que-nacen-menos-ninos-fines-semana-dias-festivos_2023011163bf339d8db6350001eeb36b.html\#}:\textasciitilde:text=Los\%20días\%20festivos\%20y\%20fines,el\%20de\%20Nochebuena\%20o\%20Navidad.
\item
  \url{https://pubmed.ncbi.nlm.nih.gov/17891531/\#}:\textasciitilde:text=While\%20most\%20of\%20this\%20weekend,(up\%20to\%20-14.5\%25).
\end{enumerate}

El primero es un artículo de ``La SextaClave'' que explica el por qué
hay menos niños nacidos los festivos y fines de semana en España
-\textgreater{} La causa parece estar en el número de partos inducidos
al tener en domingos y festivos un menor número de trabajadores se
tienen menos partos inducidos y por tanto menos nacimientos en domingos.
Casualmente en España a mediados de los 60 y durante la década de los 70
se empezo a tener un mayor numero de partos inducidos por los avances en
los centros hospitalarios.

El segundo articulo hace referencia a la situacion en Suiza y comenta
que, aunque sigue habiendo un menor número de nacimientos los domingos,
parece haber cierta tendencia a que el numero de nacimientos decrezca
aún más los Sábados.

\#Estudio con la estación del año. De nuevo, la hipótesis de partida
será que \textbf{NO} hay diferencia entre las distintas estaciones del
año.

\begin{Shaded}
\begin{Highlighting}[]
\NormalTok{getSeason }\OtherTok{\textless{}{-}} \ControlFlowTok{function}\NormalTok{(DATES) \{}
\NormalTok{    WS }\OtherTok{\textless{}{-}} \FunctionTok{as.Date}\NormalTok{(}\StringTok{"2012{-}12{-}15"}\NormalTok{, }\AttributeTok{format =} \StringTok{"\%Y{-}\%m{-}\%d"}\NormalTok{) }\CommentTok{\# Winter Solstice}
\NormalTok{    SE }\OtherTok{\textless{}{-}} \FunctionTok{as.Date}\NormalTok{(}\StringTok{"2012{-}3{-}15"}\NormalTok{,  }\AttributeTok{format =} \StringTok{"\%Y{-}\%m{-}\%d"}\NormalTok{) }\CommentTok{\# Spring Equinox}
\NormalTok{    SS }\OtherTok{\textless{}{-}} \FunctionTok{as.Date}\NormalTok{(}\StringTok{"2012{-}6{-}15"}\NormalTok{,  }\AttributeTok{format =} \StringTok{"\%Y{-}\%m{-}\%d"}\NormalTok{) }\CommentTok{\# Summer Solstice}
\NormalTok{    FE }\OtherTok{\textless{}{-}} \FunctionTok{as.Date}\NormalTok{(}\StringTok{"2012{-}9{-}15"}\NormalTok{,  }\AttributeTok{format =} \StringTok{"\%Y{-}\%m{-}\%d"}\NormalTok{) }\CommentTok{\# Fall Equinox}

    \CommentTok{\# Convierte todos los años a una fecha en 2012}
    \CommentTok{\#2012 es un buen año para hacer esto porque es bisiesto}
\NormalTok{    d }\OtherTok{\textless{}{-}} \FunctionTok{as.Date}\NormalTok{(}\FunctionTok{strftime}\NormalTok{(DATES, }\AttributeTok{format=}\StringTok{"2012{-}\%m{-}\%d"}\NormalTok{))}

    \FunctionTok{ifelse}\NormalTok{ (d }\SpecialCharTok{\textgreater{}=}\NormalTok{ WS }\SpecialCharTok{|}\NormalTok{ d }\SpecialCharTok{\textless{}}\NormalTok{ SE, }\DecValTok{1}\NormalTok{, }\CommentTok{\#1 = Winter}
      \FunctionTok{ifelse}\NormalTok{ (d }\SpecialCharTok{\textgreater{}=}\NormalTok{ SE }\SpecialCharTok{\&}\NormalTok{ d }\SpecialCharTok{\textless{}}\NormalTok{ SS, }\DecValTok{2}\NormalTok{,}\CommentTok{\#2 = Spring}
        \FunctionTok{ifelse}\NormalTok{ (d }\SpecialCharTok{\textgreater{}=}\NormalTok{ SS }\SpecialCharTok{\&}\NormalTok{ d }\SpecialCharTok{\textless{}}\NormalTok{ FE, }\DecValTok{3}\NormalTok{, }\DecValTok{4}\NormalTok{))) }\CommentTok{\#3 = Summer, 4 = Fall}
\NormalTok{\}}

\NormalTok{seasons }\OtherTok{\textless{}{-}} \FunctionTok{getSeason}\NormalTok{(}\FunctionTok{as.Date}\NormalTok{(nacFLr}\SpecialCharTok{$}\NormalTok{Fecha,}\StringTok{"\%Y{-}\%m{-}\%d"}\NormalTok{))}
\NormalTok{nacFLrd}\OtherTok{=}\FunctionTok{cbind}\NormalTok{(nacFLrdl,seasons)}


\FunctionTok{plot}\NormalTok{(nacFLrd}\SpecialCharTok{$}\NormalTok{seasons, nacFLrd}\SpecialCharTok{$}\NormalTok{Numero\_de\_nacimientos)}
\FunctionTok{abline}\NormalTok{(}\FunctionTok{lm}\NormalTok{(nacFLrd}\SpecialCharTok{$}\NormalTok{Numero\_de\_nacimientos}\SpecialCharTok{\textasciitilde{}}\NormalTok{nacFLrd}\SpecialCharTok{$}\NormalTok{seasons))}
\end{Highlighting}
\end{Shaded}

\includegraphics{Nacimientos_files/figure-latex/unnamed-chunk-12-1.pdf}

\begin{Shaded}
\begin{Highlighting}[]
\NormalTok{Pearson }\OtherTok{=} \FunctionTok{cor}\NormalTok{(nacFLrd}\SpecialCharTok{$}\NormalTok{Numero\_de\_nacimientos,nacFLrd}\SpecialCharTok{$}\NormalTok{seasons)}
\NormalTok{Pearson}
\end{Highlighting}
\end{Shaded}

\begin{verbatim}
## [1] 0.2864479
\end{verbatim}

Tanto en la gráfica como en el coeficiente de linealidad de Pearson
parece haber una dependencia positiva. Más niños nacen a final de año
que a principio del mismo. En el siguiente link hay información de
interés:
\url{https://www.livescience.com/32728-baby-month-is-almost-here-.html\#}:\textasciitilde:text=In\%20the\%20United\%20States\%2C\%20the,Health\%20Statistics\%2C\%20told\%20Live\%20Science.

Parece ser que es en Agosto y en Julio (Verano) cuando más bebés nacen
en Estados Unidos -\textgreater{}

``A 1990 study in the Journal of Biological Rhythms(opens in new tab)
suggested the seasonality of human births may be linked with local
temperature and day length. The extent to which temperature and day
length may or may not change seasonally over the year depends in part on
latitude, Martinez and Bakker noted. These environmental changes may
influence the frequency of sex or how fertile men or women are, they
noted. However, in their 2014 study, they note many other factors may
play a role as well, such as income, culture, holidays and rainfall,
making it challenging discussing whether and in what way temperature or
day length might affect human births.''

Sin embargo, Florida además de no tener un clima demasiado cambiante con
la época del año, es uno de los estados con mayor poder adquisitivo de
estados unidos. De hecho, según
\url{https://www.lanacion.com.ar/estados-unidos/por-que-las-personas-de-mayor-poder-adquisitivo-en-estados-unidos-eligen-mudarse-al-famoso-sun-belt-nid16082022/}
es el estado con mayor número de hogares con altos ingresos de Estados
Unidos por lo que no deberíamos ver demasiado cambio en la natalidad y
en caso de haberlo no deberían tener sus causas en lo que se explica en
el artículo anteriormente citado.

\begin{Shaded}
\begin{Highlighting}[]
\NormalTok{acumdcl}\OtherTok{=}\DecValTok{1}\SpecialCharTok{:}\DecValTok{4}
\ControlFlowTok{for}\NormalTok{(i }\ControlFlowTok{in} \DecValTok{1}\SpecialCharTok{:}\DecValTok{4}\NormalTok{)\{acumdcl[i]}\OtherTok{=}\FunctionTok{sum}\NormalTok{(nacFLrd}\SpecialCharTok{$}\NormalTok{Numero\_de\_nacimientos[}\FunctionTok{which}\NormalTok{(nacFLrd}\SpecialCharTok{$}\NormalTok{seasons}\SpecialCharTok{==}\NormalTok{i)])\}}
\FunctionTok{barplot}\NormalTok{(acumdcl, }\AttributeTok{name =} \FunctionTok{c}\NormalTok{(}\StringTok{"Winter"}\NormalTok{, }\StringTok{"Spring"}\NormalTok{, }\StringTok{"Summer"}\NormalTok{, }\StringTok{"Fall"}\NormalTok{), }\AttributeTok{col=}\StringTok{"lightgreen"}\NormalTok{)}
\end{Highlighting}
\end{Shaded}

\includegraphics{Nacimientos_files/figure-latex/unnamed-chunk-14-1.pdf}
Efectivamente, no parece haber un mayor número de nacimientos en los
meses de verano.

\hypertarget{anuxe1lisis-de-los-meses.}{%
\subsection{Análisis de los meses.}\label{anuxe1lisis-de-los-meses.}}

Según los datos del articulo citado anteriormente deberían ser los meses
de Agosto y Julio los que tengan un mayor número de nacimientos. Sin
embargo seguimos siendo escépticos y nuestra hipótesis de partida será
que \textbf{NO} hay diferencia de nacimiento entre los distintos meses
del año

\begin{Shaded}
\begin{Highlighting}[]
\FunctionTok{plot}\NormalTok{(nacFL}\SpecialCharTok{$}\NormalTok{Mes, nacFL}\SpecialCharTok{$}\NormalTok{Numero\_de\_nacimientos)}
\FunctionTok{abline}\NormalTok{(}\FunctionTok{lm}\NormalTok{(nacFL}\SpecialCharTok{$}\NormalTok{Numero\_de\_nacimientos}\SpecialCharTok{\textasciitilde{}}\NormalTok{nacFL}\SpecialCharTok{$}\NormalTok{Mes))}
\end{Highlighting}
\end{Shaded}

\includegraphics{Nacimientos_files/figure-latex/unnamed-chunk-15-1.pdf}

\begin{Shaded}
\begin{Highlighting}[]
\NormalTok{acumdcl}\OtherTok{=}\DecValTok{1}\SpecialCharTok{:}\DecValTok{12}
\ControlFlowTok{for}\NormalTok{(i }\ControlFlowTok{in} \DecValTok{1}\SpecialCharTok{:}\DecValTok{12}\NormalTok{)\{acumdcl[i]}\OtherTok{=}\FunctionTok{sum}\NormalTok{(nacFL}\SpecialCharTok{$}\NormalTok{Numero\_de\_nacimientos[}\FunctionTok{which}\NormalTok{(nacFL}\SpecialCharTok{$}\NormalTok{Mes}\SpecialCharTok{==}\NormalTok{i)])\}}
\FunctionTok{barplot}\NormalTok{(acumdcl, }\AttributeTok{name =} \FunctionTok{c}\NormalTok{(}\StringTok{"Enero"}\NormalTok{, }\StringTok{"Febrero"}\NormalTok{, }\StringTok{"Marzo"}\NormalTok{, }\StringTok{"Abril"}\NormalTok{, }\StringTok{"Mayo"}\NormalTok{, }\StringTok{"Junio"}\NormalTok{, }
                          \StringTok{"Julio"}\NormalTok{, }\StringTok{"Agosto"}\NormalTok{, }\StringTok{"Septiembre"}\NormalTok{, }\StringTok{"Octubre"}\NormalTok{, }\StringTok{"Noviembre"}\NormalTok{, }\StringTok{"Diciembre"}\NormalTok{),}\AttributeTok{col=}\StringTok{"lightblue"}\NormalTok{, }\AttributeTok{las =} \DecValTok{2}\NormalTok{)}
\end{Highlighting}
\end{Shaded}

\includegraphics{Nacimientos_files/figure-latex/unnamed-chunk-16-1.pdf}
Pareciera ser que en enero el número de nacimientos en enero es
superior.

\begin{Shaded}
\begin{Highlighting}[]
\CommentTok{\# Media de los meses}
\NormalTok{admes }\OtherTok{=}\NormalTok{acumdcl[}\DecValTok{1}\SpecialCharTok{:}\DecValTok{12}\NormalTok{]}
\NormalTok{medi}\OtherTok{=}\FunctionTok{sum}\NormalTok{(admes)}\SpecialCharTok{/}\DecValTok{12}
\CommentTok{\# Enero vs Media}
\NormalTok{(acumdcl[}\DecValTok{1}\NormalTok{]}\SpecialCharTok{/}\NormalTok{medi}\DecValTok{{-}1}\NormalTok{)}\SpecialCharTok{*}\DecValTok{100} 
\end{Highlighting}
\end{Shaded}

\begin{verbatim}
## [1] 16.81446
\end{verbatim}

\begin{Shaded}
\begin{Highlighting}[]
\CommentTok{\#Julio vs Media}
\NormalTok{(acumdcl[}\DecValTok{7}\NormalTok{]}\SpecialCharTok{/}\NormalTok{medi}\DecValTok{{-}1}\NormalTok{)}\SpecialCharTok{*}\DecValTok{100} 
\end{Highlighting}
\end{Shaded}

\begin{verbatim}
## [1] -10.45722
\end{verbatim}

En el mes de enero parece haber un 16.81 \% más de nacimientos que la
meda. Mientras que en el mes de julio hay un 10\% menos de nacimientos

\end{document}
